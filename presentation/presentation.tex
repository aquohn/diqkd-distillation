% Copyright 2016 by Wang Kunzhen <wangkunzhen1993@gmail.com>.
%
% This is a latex template adapted from Till Tantau's Beamer template.
% It adds theme customizations for the convenience of users from the
% National University of Singapore. 
% 
% In principle, this file can be redistributed and/or modified under
% the terms of the GNU Public License, version 2.
%
% However, this file is supposed to be a template to be modified
% for your own needs. For this reason, if you use this file as a
% template and not specifically distribute it as part of a another
% package/program, I grant the extra permission to freely copy and
% modify this file as you see fit and even to delete this copyright
% notice. 

\documentclass[xcolor=dvipsnames]{beamer}

% There are many different themes available for Beamer. A comprehensive
% list with examples is given here:
% http://deic.uab.es/~iblanes/beamer_gallery/index_by_theme.html
% You can uncomment the themes below if you would like to use a different
% one:
%\usetheme{AnnArbor}
%\usetheme{Antibes}
%\usetheme{Bergen}
% \usetheme{Berkeley}
%\usetheme{Berlin}
%\usetheme{Boadilla}
% \usetheme{boxes}
%\usetheme{CambridgeUS}
%\usetheme{Copenhagen}
%\usetheme{Darmstadt}
%\usetheme{default}
\usetheme{Frankfurt}
%\usetheme{Goettingen}
%\usetheme{Hannover}
% \usetheme{Ilmenau}
% \usetheme{JuanLesPins}
% \usetheme{Luebeck}
% \usetheme{Madrid}
% \usetheme{Malmoe}
%\usetheme{Marburg}
% \usetheme{Montpellier}
% \usetheme{PaloAlto}
% \usetheme{Pittsburgh}
% \usetheme{Rochester}
% \usetheme{Singapore}
% \usetheme{Szeged}
% \usetheme{Warsaw}

\definecolor{nus-orange}{RGB}{239,124,0} 
\definecolor{nus-white}{RGB}{255,255,255}
\definecolor{nus-blue}{RGB}{0,61,124}
\definecolor{nus-black}{RGB}{0,0,0}

% Uncomment this section if you want the title background for each slide to be gradient like decaying from nus-orange to nus-white.
% \useoutertheme{shadow}
% \usepackage{tikz}
% \usetikzlibrary{shadings}
% \colorlet{titleleft}{nus-orange}
% \colorlet{titleright}{nus-orange!45!nus-white}
% \makeatletter
% \pgfdeclarehorizontalshading[titleleft,titleright]{beamer@frametitleshade}{\paperheight}{%
%   color(0pt)=(titleleft);
%   color(\paperwidth)=(titleright)}
% \makeatother
% End of gradient slide title effect.

\setbeamercolor{section in head/foot}{bg=nus-blue, fg=nus-white}
\setbeamercolor{subsection in head/foot}{bg=nus-blue, fg=nus-white}
\setbeamercolor{frametitle}{bg=nus-orange, fg=nus-white}
\setbeamercolor{title}{bg=nus-orange, fg=nus-white}
\setbeamercolor{alerted text}{fg=nus-orange}
\setbeamercolor{block title}{fg=nus-white}
\setbeamercolor{block body}{fg=nus-black}

\setbeamertemplate{theorems}[numbered]
\setbeamertemplate{propositions}[numbered]

\setbeamertemplate{bibliography item}{\insertbiblabel}

\setbeamertemplate{title page}[default][colsep=-4bp,rounded=true, shadow=true]

\usefonttheme[onlymath]{serif}
\setbeamertemplate{footline}[totalframenumber]

\addtobeamertemplate{navigation symbols}{}{
  \hspace{1em} \raisebox{1.35pt}{\usebeamerfont{footline}%
\insertframenumber/\inserttotalframenumber} }

\renewcommand\qedsymbol{$\blacksquare$}
% \setbeamertemplate{qed symbol}{$\blacksquare$}

\usepackage{braket}
\usepackage{multirow}
\usepackage{adjustbox} % allow tables to take up the space they need
\usepackage{graphicx} % allows figures to be inserted and scaled
\usepackage{listings} % allows for source code blocks
\usepackage{fancyvrb} % robust verbatim

\usepackage[backend=biber, style=ieee]{biblatex}  
\addbibresource{../wiring.bib}
% \usepackage{hyperref} % hyperlinks
% \hypersetup{%
%   colorlinks=true,
%   linkcolor=purple,
%   urlcolor=blue
% }
\graphicspath{ {../images/} }

\newcommand{\concat}{\mathbin{\Vert}} % string concatenation operator
\newcommand{\rep}{\stackrel{r}{=}} % enable stacked r and = for representation
\newcommand{\?}{\mathrel{?}} % ternary operator

\DeclareMathOperator*{\argmin}{argmin}
\DeclareMathOperator*{\argmax}{argmax}
\newcommand{\norm}[1]{\left\lVert#1\right\rVert} % define norm
\newcommand{\abs}[1]{\left\lvert#1\right\rvert} % define abs
\newcommand{\Dif}{\mathop{}\!\mathrm{D}} % Difference operator
\newcommand{\dif}{\mathop{}\!\mathrm{d}} % differential d
\newcommand{\ceil}[1]{\left\lceil#1\right\rceil} % enables \ceil{} for ceil delimiter
\newcommand{\floor}[1]{\left\lfloor#1\right\rfloor} % enables \floor{} for floor delimiter
\newcommand{\cvec}[1]{\boldsymbol{\mathbf{#1}}}    % shortcut for column vectors
\newcommand{\rvec}[1]{\boldsymbol{\mathbf{#1}}^{T}} % shortcut for transposed row vectors
\newcommand{\Z}{\mathbb{Z}} % for integers
\newcommand{\R}{\mathbb{R}} % for reals
\newcommand{\C}{\mathbb{C}} % for complex
\newcommand{\dintv}[2]{\left\{#1,\ldots,#2\right\}}
\newcommand{\ocintv}[2]{\left(#1,#2\right]}
\newcommand{\cointv}[2]{\left[#1,#2\right)}
\newcommand{\ccintv}[2]{\left[#1,#2\right]}
\newcommand{\oointv}[2]{\left(#1,#2\right)}
\newcommand{\matr}[1]{\left[\mathbf{#1}\right]} % shortcut for matrices
\newcommand{\matrp}[2]{\left[\mathbf{#1}#2\right]} % shortcut for matrices with subscripts/superscripts
\newcommand{\rv}[1]{\boldsymbol{\mathbf{#1}}} % random variable
\newcommand{\Tr}{\mathrm{Tr}} % enables trace operator
\newcommand{\id}{\mathrm{id}} % enables id operator
\newcommand{\E}{\mathbb{E}} % expectation
\newcommand{\angleb}[1]{\left\langle #1 \right\rangle} % physicist's notation for mean

\newenvironment{Array}[1] % less cramped display mode arrays
{\def\arraystretch{1.75}\everymath={\displaystyle}\[\begin{array}{#1}}
{\end{array}\]}

\usepackage{mathtools}
\usepackage{tikz}
\usepackage{tikzscale} % include .tikz files with includegraphics and scale them
\usetikzlibrary{shapes, shapes.geometric, automata, positioning, arrows.meta, decorations.markings,decorations.pathreplacing, intersections, math, 3d, backgrounds}
\newcommand{\expandidx}[2]{%
  \expandafter#1\expandafter{\the\numexpr#2\relax}%
}

\tikzset{%
  >={Stealth}, % makes the arrow heads bold
  node distance=3cm, % specifies the minimum distance between two nodes. Change if necessary.
  every state/.style={thick, fill=gray!10}, % sets the properties for each ’state’ node
  initial text=$ $, % sets the text that appears on the start arrow
  % style to apply some styles to each segment of a path
  on each segment/.style={
    decorate,
    decoration={
      show path construction,
      moveto code={},
      lineto code={
        \path [#1]
        (\tikzinputsegmentfirst) -- (\tikzinputsegmentlast);
      },
      curveto code={
        \path [#1] (\tikzinputsegmentfirst)
        .. controls
        (\tikzinputsegmentsupporta) and (\tikzinputsegmentsupportb)
        ..
        (\tikzinputsegmentlast);
      },
      closepath code={
        \path [#1]
        (\tikzinputsegmentfirst) -- (\tikzinputsegmentlast);
      },
    },
  },
  % style to add an arrow in the middle of a path
  mid arrow/.style={postaction={decorate,decoration={
        markings,
        mark=at position .5 with {\arrow[#1]{stealth}}
  }}},
}


\tikzset{%
  % simple black circle at node
  point/.style={circle,draw,inner sep=0pt,minimum size=3pt,fill=black},
  % coin with label and colours
  % TODO DOES NOT WORK
  set coin/.code={\pgfqkeys{/tikz/coin}{#1}},
  set coin={bg/.initial=black, fg/.initial=white},
  coin/.style={
    set coin={#1},
    postaction={
      circle,draw=\pgfkeysvalueof{/tikz/coin/fg},
      text=\pgfkeysvalueof{/tikz/coin/fg},
      fill=\pgfkeysvalueof{/tikz/coin/bg}
    }
  }
}

\RequirePackage{luatex85}
% Default preamble
\usepackage{pgfplots}
\pgfplotsset{compat=newest}
\usepgfplotslibrary{groupplots}
\usepgfplotslibrary{polar}
\usepgfplotslibrary{smithchart}
\usepgfplotslibrary{statistics}
\usepgfplotslibrary{dateplot}
\usepgfplotslibrary{ternary}
\usetikzlibrary{arrows.meta}
\usetikzlibrary{backgrounds}
\usepgfplotslibrary{patchplots}
\usepgfplotslibrary{fillbetween}
\pgfplotsset{%
  layers/standard/.define layer set={%
    background,axis background,axis grid,axis ticks,axis lines,axis tick labels,pre main,main,axis descriptions,axis foreground%
    }{grid style= {/pgfplots/on layer=axis grid},%
    tick style= {/pgfplots/on layer=axis ticks},%
    axis line style= {/pgfplots/on layer=axis lines},%
    label style= {/pgfplots/on layer=axis descriptions},%
    legend style= {/pgfplots/on layer=axis descriptions},%
    title style= {/pgfplots/on layer=axis descriptions},%
    colorbar style= {/pgfplots/on layer=axis descriptions},%
    ticklabel style= {/pgfplots/on layer=axis tick labels},%
    axis background@ style={/pgfplots/on layer=axis background},%
    3d box foreground style={/pgfplots/on layer=axis foreground},%
  },
}

\newcommand{\frI}{\mathfrak{I}}
\newcommand{\frX}{\mathfrak{X}}
\newcommand{\rvX}{\rv{X}}
\newcommand{\frA}{\mathfrak{A}}
\newcommand{\frC}{\mathfrak{C}}
\newcommand{\frR}{\mathfrak{R}}
\newcommand{\Dist}{\mathcal{D}}
\newcommand{\DistRI}{\Dist(\frR, \frI)}
\newcommand{\Fail}{\mathcal{F}}
\newcommand{\IdealObsv}{\mathcal{I}}

\newcommand{\psubs}{p_{\rm subs}}
\newcommand{\pimpr}{p_{\rm impr}}

\newcommand{\sM}{\mathcal{M}}
\newcommand{\sS}{\mathcal{S}}
\newcommand{\sK}{\mathcal{K}}
\newcommand{\sT}{\mathcal{T}}
\newcommand{\sH}{\mathcal{H}}
\newcommand{\sV}{\mathcal{V}}
\newcommand{\sX}{\mathcal{X}}
\newcommand{\sZ}{\mathcal{Z}}
\newcommand{\cE}{\mathcal{E}}
\newcommand{\cF}{\mathcal{F}}
\newcommand{\cP}{\mathcal{P}}

\newcommand{\AU}{\mathrm{AU}_{2}}
\newcommand{\AXU}{\mathrm{AXU}_{2}}
\newcommand{\ASU}{\mathrm{ASU}_{2}}
\newcommand{\eAU}{\epsilon\text{-}\AU}
\newcommand{\eAXU}{\epsilon\text{-}\AXU}
\newcommand{\eASU}{\epsilon\text{-}\ASU}

\newcommand{\meas}{\rm meas}
\newcommand{\perm}{\rm perm}
\newcommand{\pe}{\rm pe}
\newcommand{\pa}{\rm pa}
\newcommand{\ir}{\rm ir}
\newcommand{\leakir}{\mathrm{leak}_{\ir}}
\newcommand{\auth}{\rm auth}
\newcommand{\key}{\rm key}
\newcommand{\rob}{\rm rob}
\newcommand{\cor}{\rm cor}
\newcommand{\secur}{\rm sec}
\newcommand{\erob}[1]{\epsilon_{\rob}^{(#1)}}

\newcommand{\HC}{\mathrm{HC}}
\newcommand{\MC}{\mathrm{MC}}

\newcommand{\Ls}{\mathcal{L}}
\newcommand{\Qs}{\mathcal{Q}}
\newcommand{\NSs}{\mathcal{NS}}
\newcommand{\PR}{\mathrm{PR}}
\newcommand{\sWB}{\mathcal{WB}}
\newcommand{\sk}{\rm sk}
\newcommand{\DW}{\rm DW}
\newcommand{\std}{\rm std}
\newcommand{\crit}{\rm crit}

\setbeameroption{hide notes} % Only slides
% \setbeameroption{show only notes} % Only notes
% \setbeameroption{show notes on second screen=right} % Both
\setbeamertemplate{note page}[plain]

\title{Device-independent quantum key distribution with local wiring}

\author{John Khoo}

\institute[National University of Singapore]
{
  Department of Electrical and Computer Engineering \\ \& Department of Computer Science \\
  National University of Singapore
}

\titlegraphic{
  \includegraphics[width=2cm]{nus-logo}
}

\date{\today}

% Uncomment this, if you want the table of contents to pop up at
% the beginning of each subsection:
\AtBeginSubsection[]
{
  \begin{frame}<beamer>{Outline}
    \tableofcontents[currentsection,currentsubsection]
  \end{frame}
}

\begin{document}

\begin{frame}
  \titlepage
\end{frame}

\begin{frame}{Outline}
  \tableofcontents
\end{frame}

\section*{Introduction}
\begin{frame}{Quantum Key Distribution (QKD)}
  \begin{itemize}[<+->]
    \item Quantum processes are \alert{fundamentally indeterministic}
    \item This can be used for \alert{cryptographic purposes}
    \item Quantum correlations can be \alert{guaranteed to be private}
    \item We can \alert{distill secret keys} from them
  \end{itemize}
\end{frame}

\note{
  Like good engineers, the first thing we try to do is figure out how we can use this.
}

\begin{frame}{Device-Independent Quantum Key Distribution (DIQKD)}
  \begin{itemize}[<+->]
    \item Many \alert{assumptions} about devices needed for QKD security
    \item Impossible to rule out \alert{all possible device flaws}
    \item DIQKD:\ certify privacy using \alert{device statistics} from \alert{Bell tests}
  \end{itemize}
\end{frame}

\note{
  While our protocols may be secure in theory, the devices we choose to implement them may violate the assumptions of our proofs.

  The behaviour is fully parametrised by a small set of easily measurable variables.
}

\begin{frame}{Wiring}
  \begin{itemize}[<+->]
    \item Connect multiple boxes together with wires
    \item Overall system \alert{can still be used for DIQKD}
    \item Will have different statistics
    \item \alert{Can this be helpful for DIQKD?}
  \end{itemize}
\end{frame}

\begin{frame}{Objective}
  \begin{itemize}[<+->]
    \item The dream: \alert{efficient algorithm} to find \alert{best wiring}
    \item If not possible, \alert{find out why}
    \item Interim presentation: review approaches and techniques tried
  \end{itemize}
\end{frame}

\section{Preliminaries}

\subsection{Review}

\begin{frame}{Bell Tests}
  \begin{onlyenv}<1>
    \begin{figure}
      \centering
      \begin{tikzpicture}
        \tikzmath{\rows = 5; \cols = 3; \intra = 2; \inter = 2; \rheight = 1;
        \nrowv = \rheight+\rheight; \theight = \rows*\rheight; 
        \colm = \cols-1; \ncolh = \inter+\intra; \twidth = \colm*\inter+\cols*\intra;}
        \draw (0,0) node {Alice} ++(\intra,0) node {Bob} ++(\inter,0) node {Alice} ++(\intra,0) node {Bob} ++(\inter,0) node {Alice} ++(\intra,0) node {Bob};
        \foreach \v in {\rheight,\nrowv,...,\theight} {
          \foreach \h in {0,\ncolh,...,\twidth} {
            \tikzmath{int \a, \b, \x, \y;
              \a = random(0,1); \b = random(0,1); \x = random(0,1); \y = random(0,1);
              if \a == 1 then { let \aval = H; } else { let \aval = T; };
              if \b == 1 then { let \bval = H; } else { let \bval = T; };
              if \x == 1 then { let \afg = black; let \abg = white; } else { let \afg = white; let \abg = black; };
              if \y == 1 then { let \bfg = black; let \bbg = white; } else { let \bfg = white; let \bbg = black; };
            }

            % TODO DOES NOT WORK
            % \draw (\h, \v) node[coin={bg = \abg, fg = \afg}] {\aval} ++(\intra,0) node[coin={bg = \bbg, fg = \bfg}] {\bval};
            \draw (\h, \v) node[circle,draw=\afg, text=\afg, fill=\abg,] {\aval} ++(\intra,0) node[circle,draw=\bfg, text=\bfg, fill=\bbg, ] {\bval}; 
          }
        }

        \tikzmath{\ilineh = \intra+0.5*\inter; \nlineh = \ilineh+\intra+\inter;}
        \foreach \h in {\ilineh, \nlineh} {
          \path[draw=black] (\h, -0.5*\rheight) -- (\h, \theight+0.5*\rheight);
        }
      \end{tikzpicture}
      \caption{Uncorrelated statistics}%
    \end{figure}
  \end{onlyenv}

  \begin{onlyenv}<2>
    \begin{figure}
      \centering
      \begin{tikzpicture}
        \tikzmath{\rows = 5; \cols = 3; \intra = 2; \inter = 2; \rheight = 1;
        \nrowv = \rheight+\rheight; \theight = \rows*\rheight; 
        \colm = \cols-1; \ncolh = \inter+\intra; \twidth = \colm*\inter+\cols*\intra;}
        \draw (0,0) node {Alice} ++(\intra,0) node {Bob} ++(\inter,0) node {Alice} ++(\intra,0) node {Bob} ++(\inter,0) node {Alice} ++(\intra,0) node {Bob};
        \foreach \v in {\rheight,\nrowv,...,\theight} {
          \foreach \h in {0,\ncolh,...,\twidth} {
            \tikzmath{int \a, \b, \x, \y;
              \x = random(0,1); \y = random(0,1);
              if \x == 1 then { let \afg = black; let \abg = white; } else { let \afg = white; let \abg = black; };
              if \y == 1 then { let \bfg = black; let \bbg = white; } else { let \bfg = white; let \bbg = black; };
            }

            \draw (\h, \v) node[circle,draw=\afg, text=\afg, fill=\abg,] {H} ++(\intra,0) node[circle,draw=\bfg, text=\bfg, fill=\bbg, ] {H}; 
          }
        }

        \tikzmath{\ilineh = \intra+0.5*\inter; \nlineh = \ilineh+\intra+\inter;}
        \foreach \h in {\ilineh, \nlineh} {
          \path[draw=black] (\h, -0.5*\rheight) -- (\h, \theight+0.5*\rheight);
        }
      \end{tikzpicture}
      \caption{Locally correlated statistics}%
    \end{figure}
  \end{onlyenv}

  \begin{onlyenv}<3>
    \begin{figure}
      \centering
      \begin{tikzpicture}
        \tikzmath{\rows = 5; \cols = 3; \intra = 2; \inter = 2; \rheight = 1;
        \nrowv = \rheight+\rheight; \theight = \rows*\rheight; 
        \colm = \cols-1; \ncolh = \inter+\intra; \twidth = \colm*\inter+\cols*\intra;}
        \draw (0,0) node {Alice} ++(\intra,0) node {Bob} ++(\inter,0) node {Alice} ++(\intra,0) node {Bob} ++(\inter,0) node {Alice} ++(\intra,0) node {Bob};
        \foreach \v in {\rheight,\nrowv,...,\theight} {
          \foreach \h in {0,\ncolh,...,\twidth} {
            \tikzmath{int \a, \b, \x, \y;
              \x = random(0,1); \y = random(0,1); \o = random(0,1);
              if \o == 1 then { let \aval = H; let \bval = T; } else { let \aval = T; let \bval = H; };
              if \x == 1 then { let \afg = black; let \abg = white; } else { let \afg = white; let \abg = black; };
              if \y == 1 then { let \bfg = black; let \bbg = white; } else { let \bfg = white; let \bbg = black; };
              if \x * \y != 1 then { let \bval = \aval; };
            }

            \draw (\h, \v) node[circle,draw=\afg, text=\afg, fill=\abg,] {\aval} ++(\intra,0) node[circle,draw=\bfg, text=\bfg, fill=\bbg, ] {\bval}; 
          }
        }

        \tikzmath{\ilineh = \intra+0.5*\inter; \nlineh = \ilineh+\intra+\inter;}
        \foreach \h in {\ilineh, \nlineh} {
          \path[draw=black] (\h, -0.5*\rheight) -- (\h, \theight+0.5*\rheight);
        }
      \end{tikzpicture}
      \caption{Nonlocally correlated statistics}%
    \end{figure}
  \end{onlyenv}
\end{frame}

\begin{frame}{Nonlocal Statistics}
  \begin{onlyenv}<1-4>
  \begin{itemize}[<+->]
    \item Alice: input \(x\), output \(a\); Bob: input \(y\), output \(b\)
    \item Full statistics: \emph{behaviour} \(p(ab|xy)\)
    \item Must fulfill \emph{no-signalling conditions}
      \begin{gather*}
        \sum_b p(ab|xy) = p(a|xy) = p(a|x)\;\forall a,x,y \\
        \sum_a p(ab|xy) = p(b|xy) = p(b|y)\;\forall b,x,y
      \end{gather*}
    \item \emph{Setting}: \((i_A, o_A, i_B, o_B)\)
    \[ x \in \dintv{1}{i_A}, a \in \dintv{1}{o_A}, y \in \dintv{1}{i_B}, b \in \dintv{1}{o_B} \]
  \end{itemize}
  \end{onlyenv}
  \begin{onlyenv}<5->
  \begin{itemize}[<+->]
    \item Focus on \emph{binary-output} settings
    \item \emph{Correlators}:
      \[ \angleb{A_x B_y} \coloneqq p(a=b|xy) - p(a\neq b|xy) \]
    \item \emph{Quantum bit error rate} (QBER):
      \[ Q_{xy} \coloneqq p(a \neq b|xy) = \frac{1-\angleb{A_x B_y}}{2} \]
    \item \emph{Clauser-Horne-Shimony-Holt} (CHSH) function in \((2,2,2,2)\):
      \[ S \coloneqq \angleb{A_0 B_0} + \angleb{A_1 B_0} + \angleb{A_0 B_1} - \angleb{A_1 B_1} \]
  \end{itemize}
  \end{onlyenv}
\end{frame}

\begin{frame}{Nonlocal Behaviours}
  \begin{itemize}[<+->]
    \item \emph{Local set} \(\Ls\): \(p(ab|xy) = p(a|x)p(b|y)\)
    \item \emph{Quantum set} \(\Qs\): \(p(ab|xy) = \Tr\left[ \left(M_{a|x} \otimes N_{b|y}\right) \rho_{AB} \right]\)
    \item \emph{No-signalling set} \(\NSs\) defined by no-signalling conditions
    \item \(\Ls \subsetneq \Qs \subsetneq \NSs\)
    \item \(\abs{S(p_{\Ls})} \leq 2\), \(\abs{S(p_{\Qs})} \leq 2\sqrt{2}\), \(\abs{S(p_{\NSs})} \leq 4\)
  \end{itemize}
\end{frame}

\begin{frame}{DIQKD Security}
  \begin{itemize}[<+->]
    \item \(\rho_{A^n B^n E^n}\): Alice--Bob--Eve ccq state after \(n\) Bell tests
    \item DIQKD protocol \(\mathfrak{P}_{\omega}\) with public data in \(D\) and key of length \(\ell\)
      \[ \mathfrak{P}_{\omega}\left[\rho_{A^n B^n E^n}\right] \mapsto \omega_{K_A K_B E^n D} \]
    \item \emph{Universally composable} \(\epsilon\)-security
      \[ (1-\erob{n})\Delta_{\Tr}(\omega_{K_A K_B E^n D}, \tau_{K_A K_B}^{\ell} \otimes \rho_{E^n} \otimes \omega_D) \leq \epsilon \]
    \item \emph{Robustness} \(\erob{n}\), \emph{trace distance} \(\Delta_{\Tr}(\rho, \sigma) = \frac{1}{2}\norm{\rho - \sigma}_1\), \emph{ideal key} \(\tau_{K_A K_B}^{\ell}\)
      \[ \tau_{K_A K_B}^{\ell} = \frac{1}{2^{\ell}} \sum_{j=1}^{2^{\ell}} \ket{j}\bra{j}_{K_A} \otimes \ket{j}\bra{j}_{K_B} \]
  \end{itemize}
\end{frame}

\begin{frame}{Asymptotic DIQKD Security}
  \begin{onlyenv}<1-2>
    \begin{itemize}[<+->]
    \item \emph{Asymptotic key rate}
      \[ r(\mathfrak{P}_{\omega}, \rho) = \sup_{\{\ell_n\}} \lim_{n\to\infty} (1-\erob{n}) \frac{\ell_n}{n} \]
      over all sequences \(\{\ell_n\}\) with asymptotically vanishing \(\epsilon\)
    \[ \lim_{n\to\infty} (1-\erob{n})\Delta_{\Tr}(\omega_{K_A K_B E^n D}, \tau_{K_A K_B}^{\ell_n} \otimes \rho_{E^n} \otimes \omega_{D}) = 0 \]
    \item Rounds are effectively i.i.d.\ in asymptotic limit: \(\rho_{A^n B^n E^n} \approx_{n\to\infty} \rho_{ABE}^{\otimes n}\)
  \end{itemize}
  \end{onlyenv}
  \begin{onlyenv}<3->
    \begin{itemize}[<+->]
    \item qqq state \(\rho_{Q_A Q_B E}\) measured using measurement strategy \(\sM\), a set of POVMs \({\{{\{M_{a|x}\}}_{a}, {\{N_{b|y}\}}_{b}\}}_{x,y}\)
    \[ \sum_{a} M_{a|x} = I_{Q_A}\;\forall x \qquad M_{a|x} \geq 0\;\forall a, x \]
    \[ \sum_{b} N_{b|y} = I_{Q_B}\;\forall y \qquad N_{b|y} \geq 0\;\forall b, y \]
  \item Behaviour given by
      \[ p(ab|xy) = \cP(\rho, \sM) = \Tr\left[\rho_{Q_A Q_B E} \left(M_{a|x} \otimes N_{b|y} \otimes I_{E}\right) \right] \]
    \item \emph{Secret key capacity} of a behaviour \(p\)
      \[ C_{\sk}(p) = \sup_{\mathfrak{P}_{\omega}} \inf_{ \substack{
            \cP(\rho, \sM) = p
        }
      } r(\mathfrak{P}_{\omega}, \rho_{ABE}) \]
  \end{itemize}
  \end{onlyenv}
\end{frame}

\subsection{Preliminary Results}

\begin{frame}{Experimental Model}
  \begin{itemize}[<+->]
    \item Probability \(n_c\) for source to not produce anything
    \item Efficiency \(\eta\): probability that detector will detect signal
    \item Error conditions assigned to outcome \(0\)
  \end{itemize}
\end{frame}

\begin{frame}{N-AND Wiring}
  \begin{itemize}[<+->]
    \item Overall input \(x\) and overall output \(a\) for \(N\) boxes
    \item Broadcast input and AND-gate output
      \[ x_i = x\;\forall i \in \dintv{1}{i} \]
      \[ a = \prod_{i=1}^N a_i \]
  \end{itemize}
\end{frame}

\begin{frame}{Result}
    \begin{figure}
      \includegraphics[width=\linewidth]{exptplt.pdf}
    \end{figure}
\end{frame}

\section{Behaviours and Wirings}

\subsection{The QKD Setting}

\begin{frame}{Polytopes}
  \begin{itemize}[<+->]
    \item \emph{Half-space}: subset of \(\R^d\) that obeys \emph{affine constraint} \(\matr{A}\cvec{x} - \cvec{b} \leq 0\)
    \item \emph{Polytopes} are unions of half-spaces
    \item \(\NSs\) and \(\Ls\) are polytopes, \alert{\(\Qs\) is not}
    \item Polytopes completely specified by either \emph{vertices} or half-space inequalities (constraints)
  \end{itemize}
\end{frame}

\begin{frame}{NS Vertices}
  \begin{onlyenv}<1>
    \begin{figure}
      \centering
      \begin{tikzpicture}[z={(0:1)}, x={(-55:0.5)}] % set projection of unit vectors onto screen
        \tikzmath{ \raylen = 3; }
        \draw[dotted] (0,0,0) node[name=boundbehav, point, label=\(p_{\Ls}^{(0)}\)] {} -- ++(0,\raylen,0) node[name=prbehav, point, label=\(p_{\PR}\)] {};
        \begin{scope}[canvas is xz plane at y=0]
          \tikzmath{
            coordinate \L;
            int \i;
            for \i in {1,2,...,8}{
              let \lname = lbehav\i;
              { \draw (22.5+45*\i:\raylen) node[point, name=\lname, label=\(p_{\Ls}^{(\i)}\)] {} -- (prbehav); };
            };
          }
        \begin{pgfonlayer}{background}
        \fill[draw,fill=blue!20,fill opacity=0.5] (lbehav1.center) -- (lbehav2.center) -- (lbehav3.center) -- (lbehav4.center) -- (lbehav5.center) -- (lbehav6.center) -- (lbehav7.center) -- (lbehav8.center) -- (lbehav1.center);
        \end{pgfonlayer}
        \end{scope}
      \end{tikzpicture}
      \caption{Representation of behaviours that violate \(S \leq 2\), with CHSH facet in blue}%
    \end{figure}
  \end{onlyenv}

  \begin{onlyenv}<2->
    \begin{itemize}
      \item 
    \end{itemize}
  \end{onlyenv}
\end{frame}

\begin{frame}{The QKD Setting}
  \begin{itemize}[<+->]
    \item Standard DIQKD setting: \((i_A, o_A, i_B, o_B) = (2,2,3,2)\)
  \end{itemize}
\end{frame}

\subsection{Local Wirings}

\section{Key Rate Bounds}

\subsection{Upper Bounds}

\subsection{Lower Bounds}

\subsection{Nonlocality Monotones}

\section*{Conclusion}

\begin{frame}{Discussion}
  \begin{onlyenv}<1-6>
    \begin{itemize}[<+->]
      \item Information-theoretic perspective: significant difference in key expansion ratio
        \begin{itemize}
          \item Limits of authentication achievable, but effect of QKD protocol unclear
        \end{itemize}
      \item However, finite-key effects only signficant in low-keyrate regime
        \begin{itemize}
          \item Authentication keys are \(\sim 10^2\) bits
          \item May have applications in resource-constrained scenarios
          \item Require more detailed analysis and optimisation
        \end{itemize}
    \end{itemize}
  \end{onlyenv}
  \begin{onlyenv}<7-10>
    \begin{itemize}[<+->]
      \item Tradeoff in \(k_{\pa}\) seems worse for polynomial hashing
        \begin{align*}
          k_{\pa}^T &= 2l + \ceil{ 4\log n + 2\log \frac{1}{\hat{\epsilon}} } \\
          k_{\pa}^P &= 2l + 2\floor{ 3 \log n - \log l - 2\log \frac{1}{\hat{\epsilon}} } - 2
        \end{align*}
      \item However, a ``polynomial stack'' seems to fare better generally
        \begin{itemize}
          \item More efficient authentication makes up for greater hash key length
        \end{itemize}
      \item The penalty from almost-universal privacy amplification impacts key length logarithmically in \(\hat{\epsilon}\)
    \end{itemize}
  \end{onlyenv}
  \begin{onlyenv}<11->
    \begin{itemize}[<+->]
      \item Information reconciliation schemes must be carefully analysed
        \begin{itemize}
          \item \(\xi\) can be punishing for small block sizes
        \end{itemize}
      \item Measurement quality has a significant impact on finite-key and asymptotic performance
    \end{itemize}
  \end{onlyenv}
\end{frame}

\begin{frame}[c]{}
  \begin{center}
    \begin{beamercolorbox}[sep=8pt,center,shadow=true,rounded=true]{title}
      Thank you!
    \end{beamercolorbox}
  \end{center}
\end{frame}

% Placing a * after \section means it will not show in the
% outline or table of contents.

\end{document}

