\documentclass[10pt, a4paper]{article}

% Page
\usepackage{pdflscape} % make certain pages landscape
\usepackage{geometry} % set page geometry
\usepackage[indent=2em, skip=0.8em plus 0.1 em minus 0.2 em]{parskip} % set paragraph spacing

% Font and Text
\usepackage[utf8]{inputenc}
\usepackage{bbm} % blackboard bold fonts
\usepackage{lmodern} % bold teletype font
\usepackage{csquotes} % allows quoting blocks of text
\usepackage{textcomp} % allow text settings
\usepackage{xcolor} % allows for colouring of source code blocks
\usepackage{url} % better URLs
\usepackage{indentfirst} % indent the first line after a heading
\usepackage{enumitem} % gives alphabet list headers

% Notation
\usepackage{amssymb} % allows for maths mode symbols
\usepackage{amsmath} % allows for many maths mode commands
\usepackage{amsthm} % allows for QED tombstone
\usepackage{mathtools}
\allowdisplaybreaks% allow display elements to break across pages
\usepackage{siunitx} % allows SI units to be formatted
\usepackage{braket} % support Dirac notation
\usepackage{esvect} % nicer vector arrows
\usepackage[thinc]{esdiff} % derivatives 
\usepackage{esint} % integrals
\usepackage{cancel} % allow cancellation

% Tables
\usepackage{tabulary} % allows for better tables
\usepackage{longtable} % allows tables to span pages
\usepackage{makecell} % allow nice table titles
\usepackage[export]{adjustbox} % allow tables to take up the space they need, and export additional scaling stuff for graphicx
\usepackage{diagbox} % diagonal box in table
\usepackage{multirow} % cells in tables can span rows/columns
\usepackage{tablefootnote} % use \tablefootnote for footnotes in tables

% Code
\usepackage{listings} % allows for source code blocks
\usepackage{algorithm} % typeset algorithms
\usepackage{algpseudocode} % pseudocode style algorithms

% Graphics
\usepackage[justification=centering]{caption} % centres captions
\usepackage{float} % stop LaTeX from making stuff float everywhere with H
\usepackage{graphicx} % allows figures to be inserted and scaled
\usepackage{tikz} % programmatically draw stuff
\usepackage{tikzscale} % include .tikz files with includegraphics and scale them
\usetikzlibrary{shapes, shapes.geometric, automata, positioning, arrows}

\RequirePackage{luatex85}
% Default preamble
\usepackage{pgfplots}
\pgfplotsset{compat=newest}
\usepgfplotslibrary{groupplots}
\usepgfplotslibrary{polar}
\usepgfplotslibrary{smithchart}
\usepgfplotslibrary{statistics}
\usepgfplotslibrary{dateplot}
\usepgfplotslibrary{ternary}
\usetikzlibrary{arrows.meta}
\usetikzlibrary{backgrounds}
\usepgfplotslibrary{patchplots}
\usepgfplotslibrary{fillbetween}
\pgfplotsset{%
  layers/standard/.define layer set={%
    background,axis background,axis grid,axis ticks,axis lines,axis tick labels,pre main,main,axis descriptions,axis foreground%
    }{grid style= {/pgfplots/on layer=axis grid},%
    tick style= {/pgfplots/on layer=axis ticks},%
    axis line style= {/pgfplots/on layer=axis lines},%
    label style= {/pgfplots/on layer=axis descriptions},%
    legend style= {/pgfplots/on layer=axis descriptions},%
    title style= {/pgfplots/on layer=axis descriptions},%
    colorbar style= {/pgfplots/on layer=axis descriptions},%
    ticklabel style= {/pgfplots/on layer=axis tick labels},%
    axis background@ style={/pgfplots/on layer=axis background},%
    3d box foreground style={/pgfplots/on layer=axis foreground},%
  },
}

\tikzset{%
  ->, % makes the edges directed
  >=stealth, % makes the arrow heads bold
  node distance=3cm, % specifies the minimum distance between two nodes. Change if necessary.
  every state/.style={thick, fill=gray!10}, % sets the properties for each ’state’ node
  initial text=$ $, % sets the text that appears on the start arrow
}

\definecolor{codegreen}{rgb}{0,0.6,0}
\definecolor{codered}{rgb}{0.6,0,0}
\definecolor{codeblue}{rgb}{0,0,0.6}
\definecolor{codepurple}{rgb}{0.58,0,0.82}
\definecolor{backcolour}{rgb}{0.95,0.95,0.95}

\lstset{%
  numbers=left,                   % where to put the line-numbers
  stepnumber=1,                   % the step between two line-numbers.        
  numbersep=5pt,                  % how far the line-numbers are from the code
  backgroundcolor=\color{white},  % choose the background color. You must add \usepackage{color}
  showspaces=false,               % show spaces adding particular underscores
  showstringspaces=false,         % underline spaces within strings
  showtabs=false,                 % show tabs within strings adding particular underscores
  tabsize=2,                      % sets default tabsize to 2 spaces
  captionpos=b,                   % sets the caption-position to bottom
  breaklines=true,                % sets automatic line breaking
  % breakatwhitespace=true,         % sets if automatic breaks should only happen at whitespace
  upquote=true, 			% set all single quotes to straight quotes
  basicstyle={\ttfamily\small},           % hackerman teletype font
  keywordstyle={\color{codegreen}\bfseries\ttfamily}, % hackerman teletype font
  commentstyle={\color{codeblue}\textit},           % fancy italic comments
  stringstyle={\color{codered}},           % string literal highlighting
  identifierstyle={}, 
  %title=\lstname,                % show the filename of files included with \lstinputlisting;
  frame=single,			% put a frame around the source code
  aboveskip=2em, % add space before code
  inputpath={./code/}
}

% analysis and calculus
\DeclareMathOperator*{\argmin}{argmin}
\DeclareMathOperator*{\argmax}{argmax}
\newcommand{\norm}[1]{\left\lVert#1\right\rVert}
\newcommand{\abs}[1]{\left\lvert#1\right\rvert}
\newcommand{\indif}{{\mathop{}\!\mkern3mu\mathchar'26\mkern-12mu \mathrm{d}}}
\newcommand{\Dif}{\mathop{}\!\mathrm{D}} % Difference operator
\newcommand{\dif}{\mathop{}\!\mathrm{d}} % differential d

% CS, discrete maths, algebra
\DeclareMathSymbol{\lneg}{\mathord}{symbols}{"18} % tildes for negation
\newcommand{\st}{\mathrel{:}} % 'such that'
\newcommand{\?}{\mathrel{?}} % ternary operator
\newcommand{\concat}{\mathbin{\Vert}} % string concatenation operator
\newcommand{\Mod}{~\mathbf{mod}~} % for mod operator
\newcommand{\Div}{~\mathbf{div}~} % for div operator
\newcommand{\Rem}{~\mathbf{rem}~} % for rem operator
\newcommand{\Z}{\mathbb{Z}} % for integers
\newcommand{\R}{\mathbb{R}} % for reals
\newcommand{\C}{\mathbb{C}} % for complex
\newcommand{\ceil}[1]{\left\lceil#1\right\rceil} % enables \ceil{} for ceil delimiter
\newcommand{\floor}[1]{\left\lfloor#1\right\rfloor} % enables \floor{} for floor delimiter

% linear algebra
\newcommand{\cvec}[1]{\boldsymbol{\mathbf{#1}}}    % column vectors
\newcommand{\rvec}[1]{\boldsymbol{\mathbf{#1}}^{\mathrm{T}}} % row vectors (transposed from column vectors)
\newcommand{\bvec}[1]{\hat{\boldsymbol{\mathbf{#1}}}} % basis vectors
\newcommand{\matr}[1]{\left[\mathbf{#1}\right]} % matrices
\newcommand{\matrp}[2]{\left[\mathbf{#1}#2\right]} % matrices with subscripts/superscripts
\newcommand{\inner}[2]{\left\langle#1,#2\right\rangle} % inner product
% \newcommand{\dim}{\rm dim} % dimension

% probability and statistics
\newcommand{\rv}[1]{\boldsymbol{\mathbf{#1}}} % random variable
\newcommand{\E}{\mathbb{E}} % expectation
\newcommand{\angleb}[1]{\left\langle #1 \right\rangle} % physicist's notation for mean
\newcommand{\Var}{\mathrm{Var}} % variance
\newcommand{\indep}{\perp \!\!\! \perp} % independence symbol
\newcommand{\indic}[1]{\mathbbm{1}{\left\{#1\right\}}} % indicator function
\newcommand{\iid}{i.i.d.\ } % independently and identically distributed

% quantum physics
\newcommand{\tr}{\mathrm{tr}} % enables Trace operator
\newcommand{\Hs}{\mathcal{H}} % enables H for Hilbert space

% category theory
% \newcommand{\hom}{\mathrm{hom}} % collection of morphisms in a category
\newcommand{\Hom}{\mathrm{Hom}} % collection of between two objects
\newcommand{\ob}{\mathrm{ob}} % collection of objects
\newcommand{\id}{\mathrm{id}} % identity morphism

% Custom lengths
\addtolength{\jot}{0.5em} % gives spacing between align* rows

\newenvironment{Tabular}[1] % less cramped tables
{\def\arraystretch{1.75}\begin{tabular}{#1}}
{\end{tabular}}
\newenvironment{Array}[1] % less cramped display mode arrays
{\def\arraystretch{1.75}\everymath={\displaystyle}\[\begin{array}{#1}}
{\end{array}\]}
\newenvironment{displaytable}[1] % environment for a simple inline table to present some information
{\vspace\abovedisplayskip\begin{center}\begin{tabular}{#1}}
{\end{tabular}\end{center}\vspace\belowdisplayskip}

\newcommand{\tableline}[1]{\dimexpr \linewidth/#1 - 2\tabcolsep}

% temporarily change margins
\newenvironment{changemargin}[2]{% 
  \begin{list}{}{%
      \setlength{\topsep}{0pt}%
      \setlength{\leftmargin}{#1}%
      \setlength{\rightmargin}{#2}%
      \setlength{\listparindent}{\parindent}%
      \setlength{\itemindent}{\parindent}%
      \setlength{\parsep}{\parskip}%
    }%
\item[]}{\end{list}}

  \usepackage[backend=biber, style=ieee]{biblatex}
  \addbibresource{../cg4003.bib}
  \graphicspath{{../images/}}
  \usepackage{hyperref} % hyperlinks
  \hypersetup{%
    colorlinks=true,
    linkcolor=purple,
    urlcolor=blue
  }
  \usepackage{cleveref} % references are automatically of the form "Section 1" etc.
  \Crefname{subsection}{Subsection}{Subsections}

  \numberwithin{equation}{section} % eqn counter rolls over at each section

  \newcounter{stmt} % definitions, theorems and lemmas share this counter
  \theoremstyle{definition}
  \newtheorem{defn}[stmt]{Definition}
  \theoremstyle{plain}
  \newtheorem{theorem}[stmt]{Theorem}
  \theoremstyle{plain}
  \newtheorem{lemma}[stmt]{Lemma}

  \newcommand{\sM}{\mathcal{M}}
  \newcommand{\sS}{\mathcal{S}}
  \newcommand{\sK}{\mathcal{K}}
  \newcommand{\sT}{\mathcal{T}}
  \newcommand{\sH}{\mathcal{H}}
  \newcommand{\sV}{\mathcal{V}}
  \newcommand{\sX}{\mathcal{X}}
  \newcommand{\sZ}{\mathcal{Z}}
  \newcommand{\cF}{\mathcal{F}}

  \newcommand{\AU}{\mathrm{AU}_{2}}
  \newcommand{\AXU}{\mathrm{AXU}_{2}}
  \newcommand{\ASU}{\mathrm{ASU}_{2}}
  \newcommand{\eAU}{\epsilon\text{-}\AU}
  \newcommand{\eAXU}{\epsilon\text{-}\AXU}
  \newcommand{\eASU}{\epsilon\text{-}\ASU}


  \newcommand{\frI}{\mathfrak{I}}
  \newcommand{\frX}{\mathfrak{X}}
  \newcommand{\frA}{\mathfrak{A}}
  \newcommand{\frC}{\mathfrak{C}}
  \newcommand{\frR}{\mathfrak{R}}

  \newcommand{\cE}{\mathcal{E}}

  \newcommand{\meas}{\rm meas}
  \newcommand{\perm}{\rm perm}
  \newcommand{\pe}{\rm pe}
  \newcommand{\pa}{\rm pa}
  \newcommand{\ir}{\rm ir}
  \newcommand{\leakir}{\mathrm{leak}_{\ir}}
  \newcommand{\auth}{\rm auth}
  \newcommand{\key}{\rm key}
  \newcommand{\rob}{\rm rob}
  \newcommand{\cor}{\rm cor}
  \newcommand{\secur}{\rm sec}

  \title{Device-independent quantum key distribution with local wiring}
  \author{John Khoo\\ \href{mailto:john_khoo@u.nus.edu}{\texttt{john\_khoo@u.nus.edu}} \\\\ AD MAIOREM DEI GLORIAM} 

  \begin{document}

  Alice and Bob have an uncharacterised source, they randomly choose settings \(x\) and \(y\) to measure, obtaining outputs \(a\) and \(b\) respectively. They estimate \(p(a,b|x,y)\) and quantities like QBER \(p(a \neq b|\text{settings})\), Bell inequality violation e.g.\ CHSH value
  \[ S = \sum_{x=0}^1 \sum_{y=0}^1 {(-1)}^{xy} \langle A_x B_y \rangle, \langle A_x B_y \rangle \coloneqq p(a=b|x,y) - p(a\neq b|x,y) \]

  Pironio et al. in NJP:
  \[ r \geq 1 - h_2(Q) - h_2\left( \frac{1 + \sqrt{(S/2)^2-1}}{2} \right) \]
  \[ \diffp{{h_2(x)}}{x} = \frac{1}{\ln 2} \log \left(\frac{1-x}{x}\right) \]
  \begin{align*}
    \ln (2) \diffp{r}{S} &= - \left( \diffp{}{S} \frac{1 + \sqrt{(S/2)^2-1}}{2} \right) \log \left( \frac{1 - \sqrt{(S/2)^2-1}}{2} / \frac{1 + \sqrt{(S/2)^2-1}}{2} \right) \\
                 &= - \left( \frac{1}{4} \left( (S/2)^2-1 \right)^{-1/2} \diffp{}{S} \left[ (S/2)^2-1 \right] \right) \log \left( \frac{1 - \sqrt{(S/2)^2-1}}{1 + \sqrt{(S/2)^2-1}} \right) \\
                 &= \underbrace{\left( \frac{S}{ 8\sqrt{(S/2)^2-1} } \right)}_{\geq 0} \log \underbrace{ \left( \frac{1 + \sqrt{(S/2)^2-1}}{1 - \sqrt{(S/2)^2-1}} \right) }_{\geq 1} \because \sqrt{(S/2)^2-1} \geq 0 \\
                 &\geq 0
  \end{align*}

  Noisy preprocessing: add noise to \(A\), improve key rate

  Local wiring: function of two copies of devices with \(r(Q, S) = 0\) can achieve \(r \geq 0\). Either NJP is not tight, or we are doing magic. If we can show that the upper bound is 0, then we are doing magic.

  Most upper bounds are useless, because they are faithful: 0 if separable, \(\geq 0\) for all entangled.

  Farkas et al., PRL 2021: nonlocality is necessary but not sufficient for DIQKD\@. In some Werner states, they exhibit nonlocality, but no projective measurements can extract secrecy (which can produce those correlations). Eve ``forgets'' some local parts to create a uniform distribution in the nonlocal region.

  Intrinsic information \(I(A:B|E)\) is upper bound on secrecy.

  \[ \ket{\psi} = \cos \theta \ket{00} + \sin \theta \ket{11} \]

  Projective measurement with some inefficiency, or convex combination of this measurement with deterministic output and uncorrelated output.

  For box with two small boxes inside:
  \[ \langle A_x B_y \rangle' = 1 - \frac{1}{4} \left[ (1 - C_{xy}) (3 - 2m_a - 2 m_b + C_{xy} ) \right] \]
      \[ m_a = \langle A_x \rangle, m_b = \langle B_y \rangle C_{xy} = \langle A_x B_y \rangle \]
      \[ {\rm QBER} = \frac{1 - \langle A_x B_y \rangle'}{2} \]

      For \(N\) boxes,
    \[ \langle A_x B_y \rangle_{N}' = 1 - \frac{{(1-m_b)}^N + {(1-m_a)}^N}{2^{N-1}} + \frac{1}{4^{N-1}} {(1-m_a-m_b+C_{xy})}^{N} \]

    \section{Local Wirings}

    With \(N\) boxes in total, Bob couples \(n\) boxes with inputs \(\cvec{y}\) and outputs \(\cvec{b}\), and \(\breve{n}\) boxes with \(\breve{\cvec{y}}\) and outputs \(\breve{\cvec{b}}\), while Alice has the remaining \(m = N - n\) boxes with inputs \(\cvec{x}\) and outputs \(\cvec{a}\).

    In general, \(b'\) is dependent on \(\cvec{x}\), \(\cvec{a}\), \(\breve{\cvec{y}}\) and \(\breve{\cvec{b}}\) since Alice and Bob can always perform those measurements first. However, conditional probability \(P(\cvec{a}\breve{\cvec{b}}|\cvec{x}\breve{\cvec{y}})\) governing these four variables is independent of \(b'\), \(\cvec{b}\) and \(\cvec{y}\).

    \[ P'(\cvec{a}\breve{\cvec{b}}b'|\cvec{x}\breve{\cvec{y}}) = P(\cvec{a}\breve{\cvec{b}}|\cvec{x}\breve{\cvec{y}}) P'_{(\cvec{a}\breve{\cvec{b}}\cvec{x}\breve{\cvec{y}})}(b')  \]

    No signalling from Alice to Bob
    \begin{align}
      \sum_{\cvec{a}} P(\cvec{a}\breve{\cvec{b}}|\cvec{x}\breve{\cvec{y}}) P'_{(\cvec{a}\breve{\cvec{b}}\cvec{x}\breve{\cvec{y}})}(b') = P'(b'|\breve{\cvec{b}}\breve{\cvec{y}}) P(\breve{\cvec{b}}|\breve{\cvec{y}})
    \end{align}

    No signalling from Bob to Alice
    \begin{align}
      \sum_{\cvec{\breve{b}}, b'} P(\cvec{a}\breve{\cvec{b}}|\cvec{x}\breve{\cvec{y}}) P'_{(\cvec{a}\breve{\cvec{b}}\cvec{x}\breve{\cvec{y}})}(b') &= \sum_{\cvec{\breve{b}}} P(\cvec{a}\breve{\cvec{b}}|\cvec{x}\breve{\cvec{y}}) \\
                                                                                                                                                     &= P(\cvec{a}|\cvec{x})
    \end{align}

    Convex combinations do not help for CHSH\@. We assume all boxes are distributed iid; if we only care about \(Q\) and expectation values (marginals and correlators) averaged over the whole protocol we can simply take this to be equal to the empirical distribution, although we might need to be a bit more careful if we consider other quantities.

    The wiring paper may not capture the full range of dynamics. Let each player have \(c\) boxes, and let subscripts denote the order in which the boxes are used. The most general wiring for Alice would be a sequence of \(c\) functions with range \(\{1..i_A\}\)
    \begin{equation} x_1 = x \end{equation}
    \begin{equation} x_j = C_j^A(x_1, \ldots, x_{j-1}, a_1, \ldots, a_{j-1}),\,j \in \{2..c\} \end{equation}
    and a final output function with range \(\{1..o_A\}\)
    \begin{equation} a = C^A(x_1, \ldots, x_{c}, a_1, \ldots, a_{c}). \end{equation}

    Simple combinatorics gives us \(i_A^{j-1} o_A^{j-1}\) possible inputs for \(C_j^A\), so the number of possible \(C_j^A\) is \(\exp_{i_A}(i_A^{j-1} o_A^{j-1})\), and the number of possible \(C^A\) is \(\exp_{o_A}(i_A^{c} o_A^{c})\). The total number of wirings is then
    \begin{equation}
      \exp_{o^A}(i_A^c o_A^c) \prod_{j=2}^c \exp_{i_A}(i_A^{j-1} o_A^{j-1}).
    \end{equation}
    The same applies to Bob with the appropriate changes of labels \(A \mapsto B\), \(x \mapsto y\), \(a \mapsto b\). 

    Let us fix the key generating settings \(A_1\) and \(B_3\) to be AND-wired, so that all functions with \(x_1 = 1\) or \(y_1 = 3\) are fixed to the AND wiring. This gives us
    \begin{equation}
      \exp_{o^A}((i_A-1)i_A^{c-1} o_A^c) \prod_{j=2}^c \exp_{i_A}((i_A-1)i_A^{j-2} o_A^{j-1})
    \end{equation}
    wirings for Alice, with the analogous relabelling for Bob.

    \section{Quantum Model}

    The probability of no signal being produced is \(n_c\), while the efficiencies of Alice and Bob's detectors are \(\eta_A\) and \(\eta_B\) respectively. When nothing is detected, either due to no signal being produced or the detector failing, we assign the \(-1\) outcome. Therefore, the probabilities are
    \begin{align*}
      p(a|x) &= \delta_{a,-1}(n_c + 1 - \eta_A) + (\eta_A - n_c)\tilde{p}(a|x) \\
      p(b|y) &= \delta_{b,-1}(n_c + 1 - \eta_B) + (\eta_B - n_c)\tilde{p}(b|y) \\
      p(ab|xy) &= \begin{aligned}[c] \delta_{a,-1}\delta_{b,-1}n_c + (1-n_c) [ \delta_{a,-1} \delta_{b,-1}(1-\eta_A\eta_B) + \eta_A\eta_B \tilde{p}(ab|xy) \\
        + \delta_{b,-1}\eta_A(1-\eta_B)\tilde{p}(a|x) + \delta_{a,-1}\eta_B(1-\eta_A)\tilde{p}(b|y) ] 
      \end{aligned}
    \end{align*}

    For our purposes, we fix \(\eta_A = \eta_B = \eta\). Then, we have
    \begin{align*}
      \diffp{}{{n_c}} \angleb{A_x B_y} &= \eta \left( \left(1-\eta\right) \left(\langle\tilde{A}_x\rangle + \langle\tilde{B}_y\rangle\right) - \eta \langle\tilde{A}_x\tilde{B}_y\rangle + \left(2 - \eta\right) \right) \\
      \diffp{}{{\eta}} \angleb{A_x B_y} &= \left(1 - n_{c}\right) \left( \left(1-2\eta\right) \left(\langle\tilde{A}_x\rangle+\langle\tilde{B}_y\rangle\right) - 2\eta \langle\tilde{A}_x\tilde{B}_y\rangle + (2-2\eta) \right) \\
      \diffp{}{S} H^S(A|E) &= \frac{S}{4\sqrt{S^2-4}} \log_2 \left[ \frac{\sqrt{S^2-4}-2}{\sqrt{S^2-4}+2} \right]
    \end{align*}


    \section{Asymmetric CHSH}

    The lower bound for \(H(A_1|E)\), with \(s = S_{\alpha}\) from the test statistics, is
    \[ H(A_1|E) \geq g_{q,\alpha}(s) = \begin{cases}
      g(s) & \text{ for } \abs{\alpha} \geq 1 \text{ or } s \geq s^* \\
      h(q) + g'(s^*)(\abs{s}-2) & \text{ otherwise},
    \end{cases}
    \]
    where
    \begin{equation}
      g_{q,\alpha}(s) = 1 + \phi\left(R_1\right) - \phi\left(R_2\right),
    \end{equation}
    with
    \begin{equation} 
      R_1 = \sqrt{{(1-2q)}^2 + 4q(1-q)(s^2/4-\alpha^2)} \qquad R_2 = \sqrt{s^2/4-\alpha^2}
    \end{equation}
    and where \(s^*\) is the solution to
    \begin{equation}
      h(q) + g'(s^*) (s^*-2) = g(s^*).
    \end{equation}

    For a function \(R(s)\) of \(s\), it can easily be verified that
    \begin{equation}
      \diff{\phi(R)}{s} = \frac{1}{2} \diff{R}{s} \log\frac{1-R}{1+R},
    \end{equation}
    and since
    \begin{equation} 
      R_1'(s) = \frac{qs(1-q)}{R_1} \qquad R_2'(s) = \frac{s}{4R_2},
    \end{equation}
    we have
    \begin{equation}
      g_{q,\alpha}'(s) = \frac{qs(1-q)}{2R_1} \log\frac{1-R_1}{1+R_1} - \frac{s}{4R_2} \log\frac{1-R_2}{1+R_2}.
    \end{equation}


  \end{document}

